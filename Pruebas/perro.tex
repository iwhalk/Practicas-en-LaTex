documentclass[11pt]{article}

\usepackage[top=2cm, left=2cm, right=2cm, bottom=2cm]{geometry}
\usepackage{multicol}
\usepackage{graphicx}
\usepackage{caption}
\usepackage{enumitem}
\usepackage[spanish]{babel}
\usepackage{gensymb}
\usepackage{textcomp}
\usepackage{siunitx}

\title{Circuito R.L.C}
\author{Molina Escobar Carlos\\Benitez Alanis Rodrigo Moctezuma\\Ortiz Benhumea Jose Rogelio\\Jose Alexis Garcia Acundo}
\date{}

\newenvironment{Figuras}
  {\par\medskip\noindent\minipage{\linewidth}}
  {\endminipage\par\medskip}

\graphicspath{ {c:/Users/iwhalk/Documents/LaTex/Ondas_RLC/} }

\begin{document}

\maketitle

\begin{abstract}
En este ensayo se demuestran diferentes tipos de movimientos armonicos en un circuito R.L.C.
\end{abstract}


\begin{multicols}{2}

\section{CONSIDERACIONES TEÓRICAS.}
	\subsection{Resistencia.}
		Son los únicos elementos pasivos para los cuales la respuesta es la misma tanto para C. A. como para C.C. 
		Se dice que en una resistencia la tensión y la corriente están en fase.
		La resistencia es el valor de oposición al paso de la corriente (sea continua o alterna) de la resistencia.
	\subsection{Capacitor.}
		En C.C. su comportamiento es similar a las resistencias. 
		En cambio, en C.A. las señales tensión y corriente mantienen la forma de onda, pero desfasadas 90 (la corriente se adelanta $\frac{\pi}{2}$ a la tensión). 
	\subsection{Inductor.}
		En C.C. su comportamiento es similar a las resistencias.
		En cambio, en C.A. las señales tensión y corriente mantienen la forma de onda, pero desfasadas 90 (la corriente se atrasa $\frac{\pi}{2}$ con respecto a la tensión).
	\subsection{Resistencia, Capacitor e Inductor}
		La bobina y el condensador causan una oposición al paso de la corriente alterna; además de un desfase, pero idealmente no causa ninguna disipación de potencia, como si lo hace la resistencia.
		La reactancia es el valor de la oposición al paso de la corriente alterna que tienen los condensadores y las bobinas.
		Existe la reactancia capacitiva debido a los condensadores y la reactancia inductiva debido a las bobinas.
		Cuando en un mismo circuito se tienen resistencias, condensadores y bobinas y por ellas circula corriente alterna, la oposición de este conjunto de elementos al paso de la corriente alterna se llama impedancia.
		El desfase que ofrece una bobina y un condensador son opuestos, y si estos llegaran a ser de la misma magnitud, se cancelarían y la impedancia total del circuito sería igual al valor de la resistencia.
	\subsection{Circuito R.L.C.}

		circuito $R$, $L$, $C$, $2d_o$ orden con la corriente, i, como la variable independiente:

		$$L \frac{d^2i}{d^2t} + R\frac{di}{dt} + \frac{1}{C} = 0$$

		La Ecuacion resultante es: \\

		$$s^2 + s\frac{R}{L} + 1\frac{1}{LC} = 0$$

		Aa encontrar las raíces de la ecuación característica mediante el uso de la fórmula cuadrática:
		
		$$s = \frac{-R\pm \sqrt{R^2-\frac{4L}{C}} }{2L}$$

		Al sustituir las variables $\alpha$ y $\omega _0$ podemos escribir $s$ de manera un poco más sencilla como:

		$$s = - \alpha \pm \sqrt{\alpha^2 - \omega _0}$$

		Donde:

		$$\alpha = \frac{R}{2L} ; \omega _0 = \frac{1}{\sqrt{LC}}$$

		$\alpha$ se llama el factor de amortiguamiento y $\omega _0$ es la frecuencia de resonancia.
		El circuito R.C.L. es el equivalente electrónico de un péndulo con fricción. El circuito se puede modelar con esta ecuación diferencial lineal de $2d_o$. orden:

		$$L \frac{d^2i}{d^2t} + R\frac{di}{dt} + \frac{1}{C} = 0$$

		Las raíces de la ecuación característica pueden tomar formas tanto reales como complejas, dependiendo del tamaño relativo de $\alpha$ y $\omega _0$ .


		\begin{itemize}
			\item Sobre amortiguado: $\alpha > \omega _0$; conduce a la suma de dos exponenciales decrecientes.
			\item Críticamente amortiguado:  $\alpha = \omega _0$; conduce a “t” exponencial decreciente.
			\item Sub amortiguado:  $\alpha < \omega _0$; conduce a un seno decreciente.
		\end{itemize}

\section{Circuito R.L.C}

	\subsection{Esquematico del circuito}

		\begin{Figuras}
			\centering
		    \includegraphics[width=0.9\textwidth]{RLC_1}
		    \captionof{figure}{Circuito R.L.C}
		    \label{fig:mesh1}
		\end{Figuras}

		LA fiigura \ref{fig:mesh1} muestra un circuito R.L.C. sensillo, dondé, se ubican facilmente el inductor, capacito y resistencia.

	\subsection{Simulación del circuito}

		\begin{Figuras}
			\centering
		    \includegraphics[width=0.9\textwidth]{RLC_2}
		    \captionof{figure}{Los colores de las curvas coinciden con los del circuito}
		    \label{fig:mesh1}
		\end{Figuras}

		Aqui se pueden ver la diferencia de potencial de los diferentes circuito conforme avanza el tiempo, cada elemento tiene su curva caracteristica.

		\begin{Figuras}
			\centering
		    \includegraphics[width=0.9\textwidth]{RLC_3}
		    \captionof{figure}{Diferentes curvas se obtienen cambiando los parametros}
		    \label{fig:mesh1}
		\end{Figuras}
	



\end{multicols}

\end{document}