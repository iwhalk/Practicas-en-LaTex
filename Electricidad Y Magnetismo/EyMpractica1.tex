\documentclass[11pt]{article}


\title{Electroestática}
\author{2CV5\\Equipo 4\\\\Molina Escobar Carlos\\Victoria Zúñiga Mario Alberto\\Parada Velazquez Kevin Osvaldo\\Santana Martínez Jesús Gerardo\\Tovar Urrea Roberto Fausto}
\date{}

\usepackage[top=2cm, left=2cm, right=2cm, bottom=2cm]{geometry}
\usepackage{multicol}
\usepackage{graphicx}
\usepackage{enumitem}
\usepackage{caption}
\usepackage{esvect}

perrito

\newenvironment{Figuras}
  {\par\medskip\noindent\minipage{\linewidthb  {\endminipage\par\medskip}

\graphicspath{ {c:/Users/iwhalk/Documents/LaTex/EyM_imagenes/} }

\begin{document}
	\begin{multicols}{2}
	[
	\maketitle
	Los fenómenos electromagnéticos parecen ser casi imperceptibles a nuestros sentidos, es por eso que esta práctica tiene como objetivo el poder visualizar las interacciones de cuerpos 			electrizados; los fenómenos de estas interacciones de cargas puntuales y el campo eléctrico son observados consecuentemente.
	]
\section{Introduccion}
\subsection{Electroestática}
	La electrostática es la rama de la Física que estudia las interacciones entre cuerpos cargados eléctricamente que se encuentran en reposo, se ocupa de las fuerzas entre las cargas. La palabra estática significa que las cargas no se mueven, o por lo menos que no se mueven tan rápido. 
La fuerza electrostática está descrita por la ley de Coulomb. Usamos la ley de Coulomb para encontrar las fuerzas generadas por configuraciones de carga considerando que "actúa a distancia". 
\subsubsection{La carga}
La carga eléctrica es la propiedad de los objetos que da lugar a esta fuerza observada.  Las fuerzas eléctricas son muy grandes  hay dos tipos de carga eléctrica. 

\begin{Figuras}
	\centering
    \includegraphics[width=0.9\textwidth]{cargas_opuestas}
    \captionof{figure}{Las cargas opuestas se atraen}
    \label{fig:mesh1}
\end{Figuras}

\begin{Figuras}
	\centering
    \includegraphics[width=0.9\textwidth]{cargas_iguales}
    \captionof{figure}{Las cargas iguales se atraen}
    \label{fig:mesh2}
\end{Figuras}

\subsection{Ley de Coulomb}
La ley de Coulomb describe muy bien este fenómeno natural. Matemáticamente, la ley tiene la forma:


$$\vec F=K \frac{q_0 q_1}{r^2}\hat r$$

Dónde:\\

$\vec F$: : es la fuerza eléctrica, y va en dirección de la recta que une los dos cuerpos cargados.\\\\
$K$: es una constante de proporcionalidad que relaciona el lado izquierdo de la ecuación (newtons) con el lado derecho (coulombs y metros). Es necesaria para hacer que la respuesta sea correcta cuando llevamos a cabo un experimento real.\\\\
$q_0 q_1$: representan la cantidad de carga en cada cuerpo, en unidades de coulombs (la unidad del SI para la carga).\\\\
$r$: : es la distancia entre los cuerpos cargados.\\\\
$\hat r$: es un vector unitario variable que nos recuerda que la fuerza apunta en la dirección de la recta que une las dos cargas. Si las cargas son iguales, la fuerza es repulsiva; si las cargas son opuestas, la fuerza es atractiva.

\section{Material y equipo}

\begin{itemize}

	\item Péndulo Eléctrico 
	\item Paño de naylon 
	\item Paño de lana 
	\item Barra de vidrio 
	\item Barra de hierro 
	\item Electrodo de prueba 
	\item Soporte aislante 
	\item Cuba electrostática 
	\item Cables de conexión 
	\item Generador de Van de Graaft 
	\item Fuente de Alimentación 
\end{itemize}

\section{Desarrollo experimental}

\subsection{Electrización por frotamiento}

\begin{enumerate}[label=\alph*)]
	\item Frote la barra de vidrio con el paño de lana. Acérquela a algunos trocitos de papel.
	\item Nuevamente frote la barra de vidrio con el paño de lana y aproximela, sin tocar, a la esfera de sauco del péndulo eléctrico.
	\item Ahora frote la barra de poliestireno y aproxímela al péndulo eléctrico, sin tocar la esfera de sauco.\\
Repetir el procedimiento anterior con el electrodo de prueba plano en su parte metálica.
\end{enumerate}
		
\subsection{Electrización por contacto}

Tome la barra de vidrio, cargada previamente por frotamiento con el paño de lana, póngala en contacto con el electrodo de prueba plano, como se indica en la figura \ref{fig:mesh3}, y acérquelo a la esfera del péndulo eléctrico

\begin{Figuras}
	\centering
    \includegraphics[width=0.9\textwidth]{contacto}
	\captionof{figure}{}
    \label{fig:mesh3}
\end{Figuras}

\subsection{Electrización por inducción}

\begin{Figuras}
	\centering
    \includegraphics[width=0.9\textwidth]{induccion}
	\captionof{figure}{Dispositivo en cuestión}
    \label{fig:mesh4}
\end{Figuras}

Frote la barra de vidrio con el paño de lana y acérquela a la barra de metal, sin tocar, observe la esfera del péndulo eléctrico. Sin dejar de observar aleje la barra de vidrio cargada.
Repita el experimento anterior, pero ahora, antes de retirar la barra de vidrio cargada eléctricamente, toque con su dedo la barra de metal como de muestra en la figura  \ref{fig:mesh5}

\begin{Figuras}
	\centering
    \includegraphics[width=0.9\textwidth]{induccion2}
	\captionof{figure}{}
    \label{fig:mesh5}
\end{Figuras}

\subsection{Clases de Carga Eléctrica y Fuerzas de origen eléctrico}

Toque con la barra de vidrio, frotada con el paño de lana, la esfera de médula de sauco del péndulo durante un corto intervalo de tiempo.
\\

Descargue la esfera tocándola con los dedos y repita el procedimiento anterior empleando la barra de poliestireno.\\

Nuevamente cargue la esfra de medula de sauco, podiendola en contacto con la barra de vidrio, previamente frotada con el paño de lana.\\

Ahora, acerque suficientemente la esfera cargada del péndulo eléctrico, sin hacer contacto, primero la barra de vidrio cargada y depués a la barra de poliestireno frotada con el paño de lana.

\subsection{Conductores y aisladores}

\begin{Figuras}
	\centering
    \includegraphics[width=0.9\textwidth]{conductor}
	\captionof{figure}{Dispositivo en cuestión}
    \label{fig:mesh6}
\end{Figuras}

Toque un extremo de la barra de poliestirenocon la barra de vidrio cargada, previamente, por frotamiento con el paño de lana.\\

Descague la barra de vidio y colóquela en el soporte aislante, en lugar de la barra de poliestireno y cargue esta ultima repitiendo el experimento.

\subsection{Espectros del Campo Eléctrico}

Vierta el aceite de ricino en la cuba electrostática, hasta tener una capa de $4 mm$ de profundidad, Espolvoreé un poco de aserrin e instale una lenteja y arillo grande (figura \ref{fig:mesh7}), n los porta electrodos de la cuba y éstos a su vez conéctelos a tierra y a la esfera del generador Van de Graaff, repectivamente.

\begin{Figuras}
	\centering
    \includegraphics[width=0.9\textwidth]{espectro}
	\captionof{figure}{}
    \label{fig:mesh7}
\end{Figuras}

Ponga a funcionar el generador.\\

Realizado lo anterior desconecte el generador, pero antes descárguelo tocandolo, conectado previamente a tierra, y remueva por medio del agitador el aceite de ricino con el aserrin e invierta las conexiones en los portaelectrodos y póngalo a funcionar.\\

Cambie los electrodos, por otro par, de modo que se observe el campo formado por:

\begin{enumerate}[label=\alph*)]
	\item Dos cargas puntuales de diferente signo.
	\item Dos cargas puntuales del mismo signo
	\item Dos anillos circulares cargados con diferente carga
	\item Dos placas paralelas cargadas de diferente signo
	\item Un cuerpo con punta y una placa cargada con signo contrario
\end{enumerate}

\section{Resultados}

\subsection{Electrización por frotamiento}

\begin{Figuras}
	\centering
    \includegraphics[width=0.9\textwidth]{1_papel}
	\captionof{figure}{La barra atrae el papel}
    \label{fig:mesh8}
\end{Figuras}

\begin{Figuras}
	\centering
    \includegraphics[width=0.9\textwidth]{sauco}
	\captionof{figure}{La barra repele al sauco}
    \label{fig:mesh9}
\end{Figuras}

\begin{Figuras}
	\centering
    \includegraphics[width=0.9\textwidth]{1_plas}
	\captionof{figure}{La barra atrae al sauco}
    \label{fig:mesh10}
\end{Figuras}

\begin{Figuras}
	\centering
    \includegraphics[width=0.9\textwidth]{1_prueba}
	\captionof{figure}{No se aprecian efectos considerables}
    \label{fig:mesh11}
\end{Figuras}

\begin{Figuras}
	\centering
    \includegraphics[width=0.9\textwidth]{1_prueba2}
	\captionof{figure}{Se aprecian efectos minimos}
    \label{fig:mesh12}
\end{Figuras}

\subsection{Electrización por contacto}

\begin{Figuras}
	\centering
    \includegraphics[width=0.9\textwidth]{2_prueba}
	\captionof{figure}{Se aprecian efectos minimos al igual que el experimento anterior}
    \label{fig:mesh13}
\end{Figuras}

Se puede observar que la carga se reparte entre los objetos que se tocan.

\subsection{Electrización por inducción}

\begin{Figuras}
	\centering
    \includegraphics[width=0.9\textwidth]{3_vidrio}
	\captionof{figure}{Se aprecian efectos minimos pero presentes}
    \label{fig:mesh14}
\end{Figuras}

\begin{Figuras}
	\centering
    \includegraphics[width=0.9\textwidth]{3_vidrio2}
	\captionof{figure}{No se aprecian efectos}
    \label{fig:mesh15}
\end{Figuras}

Al tocar la barra esta se descarga.

\subsection{Clases de Carga Eléctrica y Fuerzas de origen eléctrico}

\begin{Figuras}
	\centering
    \includegraphics[width=0.9\textwidth]{4_vidrio}
	\captionof{figure}{Repele al sauco}
    \label{fig:mesh16}
\end{Figuras}

\begin{Figuras}
	\centering
    \includegraphics[width=0.9\textwidth]{4_plas}
	\captionof{figure}{Atrae al sauco}
    \label{fig:mesh17}
\end{Figuras}



Al igual que el experimento anterior, el que se toquen 2 objetos implica una redistribucion de sus cargas.

\subsection{Conductores y aisladores}

\begin{Figuras}
	\centering
    \includegraphics[width=0.9\textwidth]{5_plas}
	\captionof{figure}{No se preciben cambios}
    \label{fig:mesh18}
\end{Figuras}

\begin{Figuras}
	\centering
    \includegraphics[width=0.9\textwidth]{5_vidrio}
	\captionof{figure}{No se preciben cambios o muy minimos}
    \label{fig:mesh19}
\end{Figuras}

\subsection{Espectros del Campo Eléctrico}

\begin{Figuras}
	\centering
    \includegraphics[width=0.9\textwidth]{6_cp}
	\captionof{figure}{El aserrin apunta hacia la carga}
    \label{fig:mesh20}
\end{Figuras}

\begin{Figuras}
	\centering
    \includegraphics[width=0.9\textwidth]{6_bb}
	\captionof{figure}{El aserrin se aleja de ambas barras}
    \label{fig:mesh21}
\end{Figuras}

\begin{Figuras}
	\centering
    \includegraphics[width=0.9\textwidth]{6_ba}
	\captionof{figure}{El aserrin se concentra en la aguja}
    \label{fig:mesh21}
\end{Figuras}

Las distribucion del aserrin es congruente con lo contemplado en la teoria de Campos electromagneticos donde un carga pasitiva produce lineas que salen; y las negativas lineas que entran.

\section{Cuestionario}


\begin{itemize}

	\item ¿Qué es la carga eléctrica?\\
	Es una propiedad fundamental de la materia, la cual
 está asociada con partículas
atómicas, el electrón y el protón; asociando una ca
rga negativa con el electrón y
una carga positiva con el protón.
	\item ¿Cuántas clases de carga identifico en este experimento?\\
Electrización por contacto, fricción e inducción
	\item ¿Cómo se comportan las cargas eléctricas entre sí?\\
Cuando las cargas son de igual signo se repelen es 
decir que son fuerzas
electrostáticas de repulsión; cuando las cargas son de signo opuesto se atraen es
 decir que son fuerzas
electrostáticas de atracción.
	\item ¿Qué interpretación se da al principio de la co
nservación de la carga eléctrica
cuando se carga la barra de vidrio por frotamiento 
con el paño de lana?\\
 Al frotar la barra de vidrio con el paño de lana, 
este adquiere una carga positiva de
igual valor que la de la barra de vidrio, es decir,
 si la carga total antes de frotarse
la barra de vidrio y el paño era cero, después de frotarse la suma que cargas de
ambos cuerpos también será cero.
	\item ¿Cómo se podría usar una barra, cargada negativ
amente para cargar por
inducción dos barras metálicas, de manera que una quede con carga positiva y
otra con carga negativa?\\
Colocamos dos cuerpos metálicos sobre soportes de p
lástico acercamos una
varilla cargada positivamente a uno de los cuerpos,
 los electrones del metal son
atraídos por la varilla y una parte de éstos se des
plaza hacia el cuerpo dejando en
el cuerpo un déficit de electrones, es decir, una carga positiva. La carga eléctrica
de los cuerpos se ha redistribuido por inducción. Si los cuerpos se separan en
presencia de la varilla; los dos cuerpos quedan con
 cargas iguales y opuestas.
	\item ¿Cuál es la diferencia entre un conductor y un 
aislador?\\
Un conductor es aquel material que tiene la propiedad de permitir el movimiento
de cargas eléctricas mientras que un aislador impide el movimiento de estas
cargas.
	\item ¿Cómo afecta el medio ambiente a estos experimentos?\\
La temperatura puede evitar que los cuerpos se carguen con facilidad como en
casos de ambiente húmedo es difícil realizar, ya que se forma una delgada
película de humedad que evita que se genere una carga.
	\item ¿Cómo se descubrió que la carga eléctrica estab
a cuantizada?\\
a traves de la expeimentacion con 2 placas cargadas suspendiendo gotas
	\item Explique ley de coulomb\\
La magnitud de cada una de las fuerzas eléctricas con que interactúan dos cargas puntuales en reposo es directamente proporcional al producto de la magnitud de ambas cargas e inversamente proporcional al cuadrado de la distancia que las separa y tiene la dirección de la línea que las une. La fuerza es de repulsión si las cargas son de igual signo, y de atracción si son de signo contrario. 
	\item Defina los siguientes términos.
		\begin{itemize}

		\item Campo\\
Un campo de fuerza es creado por la atrac
ción y repulsión de cargas
eléctricas esto es generado por el flujo eléctrico
		\item Polaización\\
Modificación de la distribución de
 carga que ocurre en un material
aislador por efecto de un campo eléctrico.
		\item Dipolo\\
Dos cargas de signos opuestos e igual magnitud cercanas entre sí.
		\item Ionización\\
Separar los electrones de la molécula neutra través de alguna
energía como los rayo x o luz  ultravioleta.
		\item Carga puntual\\
Un pequeño espacio del cuerpo 
cargado.
		\item Gradiente de potencial\\
es el cociente resultante de dividir la variación eléctrica de
un punto $A$ menos un punto $B$ entre la variación de la distancia de los puntos hacia
la carga.
		\end{itemize}
		\item ¿Qué es una superficie equipotencial?\\
		Las superficies equipotenciales son aquellas en las
		 que el potencial toma un valor
		constante. Por ejemplo, las superficies equipotenciales creadas por cargas
		puntuales son esferas concéntricas centradas en la 
		carga, como se deduce de la
			definición de potencial$(r = cte.)$.
		\item Dos cargas, de magnitud y signos desconocidos,
 están separados una
distancia d. la intensidad del campo eléctrico es c
ero en un punto situado entre
ellas, en la línea que las une. ¿Qué se puede decir
 respecto a las cargas?\\
forman un dipolo eléctrico.
		\item Explique la ley de Gauss.
Relaciona el flujo eléctrico a través de una superficie cerrada y la
carga eléctrica encerrada por esta superficie.
\end{itemize}

\section{Conclusiones}

\subsection{Molina Escobar Carlos}

EL poder observar los fenómenos electromagnéticos de primera mano nos da una comprensión más intuitiva del tema. El efecto de los elementos cargados ya sea positiva o negativamente es perceptible con varios experimentos, así como el efecto de sus respectivos campos eléctricos.

\subsection{Parada Velazquez Kevin Osvaldo}

Lo visto en la practica me resulto muy interesante porque vimos y demostramos como es que funciona la electrostática, que a pesar de que es algo que no podemos ver, siempre está presente en nuestra vida diaria, y demostrarlo con materiales sencillos, me hace entender que estamos en constante interacción con las fuerzas y los campos eléctricos.

\subsection{Victoria Zúñiga Mario Alberto}

Gracias a los experimentos realizados  en la Práctica  Electroestatica
Pude comprende y analizar de manera más clara la parte de "cargas" el observar la fuerza ( ver el comportamiento ) que tienen cuando se atraen o se repelen. Y de ahí  demostrar  la "Ley de Coulomb".

\subsection{Santana Martínez Jesús Gerardo}

En la practica podemos concluir que muchos de los cuerpos tienden a tener carga por medio del frotamiento o simplemente contener una carga, estos son susceptibles al cambió o perturbación de su espacio.
Debido a que el campo eléctrico se encuentra presente este se puede observar por medio de experimentos muy específicos pero debido a estos nos muestran a detalle que no solamente existe perturbación sino que también existe un orden en las cosas que los rodean.

\subsection{Tovar Urrea Roberto Fausto}

En la practica se pudo concluir que los cuerpos son suceptibles a electrizarse, ya sea por frotamiento, por contacto o por induccion, y estos seran capaces de atraer o repeler otros cuerpos según si estos estas electrizados tambien o se encuentran neutrales, ademas de ello se pudo observar el espectro del campo electrico, donde se concluyo que los electrones siguen un camino de salida desde un electrodo a otro.

\section{Bibliografia}

KHAN ACADEMY. (s.f.). Electrostática | Ingeniería eléctrica | Ciencia | Khan Academy. Recuperado 17 febrero, 2020, de https://es.khanacademy.org/ science/electrical-engineering/ee-electrostatics/ee-electric-force-and-electric-field/a/ee-electric-force?modal=1\\

FISICALAB. (s.f.). Electrostática. Recuperado 17 febrero, 2020, de https://www.fisicalab.com /tema/electrostatica-intro\\

Teoría electromagnética, campos y ondas. Carl T. A. John. Limosa Noriega Editores.\\

	\end{multicols}
\end{document}