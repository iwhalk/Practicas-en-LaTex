\documentclass[11pt]{article}


\title{Distribución de las Cargas Électricas en los Conductores}
\author{2CV5\\Equipo 4\\\\Molina Escobar Carlos\\Victoria Zúñiga Mario Alberto\\Parada Velazquez Kevin Osvaldo\\Santana Martínez Jesús Gerardo\\Tovar Urrea Roberto Fausto}
\date{}

\usepackage[top=2cm, left=2cm, right=2cm, bottom=2cm]{geometry}
\usepackage{multicol}
\usepackage{graphicx}
\usepackage{caption}
\usepackage{enumitem}
\usepackage[spanish]{babel}
\usepackage{imakeidx}

\makeindex

\newenvironment{Figuras}
  {\par\medskip\noindent\minipage{\linewidth}}
  {\endminipage\par\medskip}


\graphicspath{ {c:/Users/iwhalk/Documents/LaTex/EyM_imagenes_2/} }


\begin{document}
	
\maketitle
\begin{abstract}
	Esta practica tiene como objetivo  demostrar y distinguir  por medio de experimentos (procedimientos) a los alumnos que no todos los materiales tienen la propiedad de conductividad (cuando un átomo cede con facilidad sus electrones libres).a los materiales que no son conductores se les conoce como aislantes (que no cede tan fácil sus electrones debido a que los átomos ejercen una gran atracción sobre sus ya mencionados electrones).
\end{abstract}


\begin{multicols}{2}
\section{Introduccion}
	\subsection{Objetos cargados mediante inducción}
	Es conveniente clasificar los materiales en función de la capacidad con que los electrones se mueven a través del material: Los conductores eléctricos son aquellos materiales en los cuales 		algunos de los electrones son libres, no están unidos a átomos y pueden moverse con libertad a través del material. Los aislantes eléctricos son aquellos materiales en los cuales todos los electrones 	están unidos a átomos y no pueden moverse libremente a través del material. Materiales como el vidrio, el hule y la madera seca se incluyen en la categoría de aislantes eléctricos. Cuando estos materiales son frotados, sólo la zona frotada se carga, y las partículas con carga no pueden moverse hacia otras zonas del material. En contraste, materiales como el cobre, el aluminio y la plata son buenos conductores eléctricos. Cuando están con carga en alguna pequeña zona, la carga se distribuye de inmediato en toda la superficie del material. Una tercera clase de materiales son los semiconductores, cuyas propiedades eléctricas se ubican entre las correspondientes a los aislantes y a los conductores. El silicio y el germanio son ejemplos muy conocidos de materiales semiconductores de uso común en la fabricación de una gran diversidad de chips electrónicos utilizados en computadoras, teléfonos celulares y estéreos. Las propiedades eléctricas de los semiconductores cambian, en varios órdenes de magnitud, a partir de la adición de cantidades controladas de ciertos átomos.\\
Un proceso similar a la inducción en los conductores se presenta en los materiales aislantes. En la mayoría de las moléculas neutras, el centro de la carga positiva coincide con el centro de la carga negativa. Sin embargo, en presencia de un objeto con carga, estos centros en el interior de cada molécula, en un material aislante, se desplazan ligeramente, lo que resulta en que un lado de la molécula tenga una carga más positiva que el otro. Este realineamiento de la carga en el interior de las moléculas produce una capa de carga sobre la superficie del material aislante. La proximidad de las cargas positivas en la superficie del objeto y las cargas negativas en la superficie del aislante resulta en una fuerza de atracción entre el objeto y el aislante. Su conocimiento de inducción en los materiales aislantes, le ayuda a explicar por qué una varilla cargada atrae fragmentos de papel eléctricamente neutros

\section{Material y equipo}
	\begin{itemize}

		\item Generador de Van de Graaft
		\item Banco Aislado
		\item Copa de Faraday
		\item Recipiente de plástico con esferas de cripsota
		\item Paño de lana
		\item Paño de seda
		\item Barra de vidrio
		\item Barra de poliesterina
		\item Electrodo de prueba
		\item punta de metal
		\item Fuente de Alimentación
		\item Rehilete electrostático
		\item Mechon de cabello
		\item Esfera hueca
		\item Hemisferios de Cavendish
		\item Electroscopio
		\item Cables de conexón
		\item Vela

	\end{itemize}

\section{Desarrollo experimental}

	\subsection{El electroscopio}
		 Es un dispositivo, formado por dos laminas ligerísimas, de aluminio, fijas a una varilla metálica, coronada por una esferilla también metálica. La varilla se ajusta en un tapón aislador, las dos ventanillas de cristal, una frente a la otra, permiten ver el interior.

\begin{Figuras}
	\centering
    \includegraphics[width=0.7\textwidth]{electroscopio}
	\captionof{figure}{}
    \label{fig:mesh1}
\end{Figuras}

Acerque a esfera del electroscopio una barra de vidrio sin frotar. Observe. Realizando lo anterior, cargue (frote) la barra de vidrio y acerquela hasta tocar la esfera del electroscopio.  Toque la esfera (E) con lamano y repita el procedimiento anterior con la barra de poliesterina y anote sus observaciones.\\

Con el electroscopio tocándolo con la barra de vidrio frotada con el paño de lana, de maneta que las hojas queden solo un poco separadas, acerque a la esfera, pero sin llegar a tocarla, un objeto cargado negativamente. Ahora acerque a la esfera, pero sin llegar a tocarla, un objeto cargado positivamente.  Anote lo que sucede. Por último, aproxime a la esfera, pero sin tocarla, un objeto que no haya sido frotado y que en consecuencia este probablemente descargado \\

	\subsection{La experiencia de Cavendish}
		Monte la esfera metálica hueca en el soporte aislante y colóquelo en el banco aislante. Conecta a la esfera colectora del Van der Graaf por medio del cable de conexión, teniendo cuidado que este no toque ningún otro cuerpo, como se muestra en la figura \ref{fig:mesh2}. Para cargar la esfera metálica ponga a funcionar el generador a velocidad mínima durante un minuto, aproximadamente, y apáguelo. Finalizado lo anterior desconecte la esfera metálica hueca del generador, procurando no tocar con la mano ni el generador, ni la esfera. Con la sonda prueba toque cualquier punto de la superficie de la esfera y con la ayuda del electroscopio determine si está cargada.\\

\begin{Figuras}
	\centering
    \includegraphics[width=0.9\textwidth]{cavendish}
	\captionof{figure}{}
    \label{fig:mesh2}
\end{Figuras}

Ahora tome los dos hemisferios metálicos descargados, provistos de mangos aisladores, y cubra la esfera metálica con ellos, como se muestra en la siguiente figura. Después de unos segundos separe ambos hemisferios y. Con la ayuda de la sonda prueba y del electroscopio, hemisferios. Registre sus observaciones.

\begin{Figuras}
	\centering
    \includegraphics[width=0.6\textwidth]{cavendish2}
	\captionof{figure}{}
    \label{fig:mesh3}
\end{Figuras}


	\subsection{Experiencia de Franklin}
		Instale en la parte superior de la esfera colectora del Van der Graff, previamente descargada, el recipiente con parecer de plástico, con base de metal. Ponga a funcionar a su mínima velocidad durante algunos segundos. Observe lo que sucede y regístrelo realizando lo anterior, desconecte el generador y descárguelo. Quite el recipiente con paredes de plástico del generador e instale en su lugar el cilindro metálico, (cilindro de Faraday), con las esferas conductoras, como se muestra en la figura \ref{fig:mesh4} y ponga a funcionar y generador a su máxima velocidad, durante algunos segundos.

\begin{Figuras}
	\centering
    \includegraphics[width=0.9\textwidth]{franklin}
	\captionof{figure}{}
    \label{fig:mesh4}
\end{Figuras}

	\subsection{Pantalla eléctrica}
		Coloque el capuchón metálico (G) sobre el electroscopio y conéctelo a la esfera del generador.
\begin{Figuras}
	\centering
    \includegraphics[width=0.5\textwidth]{pantalla}
	\captionof{figure}{}
    \label{fig:mesh5}
\end{Figuras}

Ponga a funcionar el generador a su máxima velocidad. 

	\subsection{Efecto de puntas}
		\begin{itemize}

		\item\textbf{Rehilete electrostático}\\
			Instale el rehilete sobre la esfera colectora del Van der Graff, ponga a funcionar en este último a su mínima velocidad. Registre sus observaciones. Si puede, aumente la velocidad

		\begin{Figuras}
			\centering
  			  \includegraphics[width=0.9\textwidth]{rehilete}
			\captionof{figure}{}
			    \label{fig:mesh6}
		\end{Figuras}

		\item\textbf{Mechon de cabello}\\
			Descargue el generador de Van der Graff, quite el rehilete y en su lugar coloque el mechón de cabellos, ponga a funcionar el generador a una velocidad media, si esta se puede regular, déjelo funcionando por espacio de un minuto.

		\item\textbf{Experiencia de la vela}\\
			Encienda la vela y ponga a funcionar el generador, acerque la flama de la vela a la punta metálica. 

		\begin{Figuras}
			\centering
  			  \includegraphics[width=0.9\textwidth]{vela}
			\captionof{figure}{}
			    \label{fig:mesh7}
		\end{Figuras}

		\end{itemize}

\section{Resultados}

	\subsection{El electroscopio}

\begin{Figuras}
	\centering
    \includegraphics[width=0.7\textwidth]{electroscopio_c}
	\captionof{figure}{De las observaciones hechas en este experimento se puede concluir que un electroscopio es un dispositivo que sirve para: determinar si un objeto pose carga eléctrica independientemente de su polaridad}
    \label{fig:mesh8}
\end{Figuras}

	\subsection{La experiencia de Cavendish}

\begin{Figuras}
	\centering
    \includegraphics[width=0.9\textwidth]{cavendish_c}
	\captionof{figure}{Se pudo observar que el electroscopio reaccionaba de manera distinta a cada hemisferio que se le acercara}
    \label{fig:mesh9}
\end{Figuras}
	\subsection{Experiencia de Franklin}
\begin{Figuras}
	\centering
    \includegraphics[width=0.9\textwidth]{franklin_c}
	\captionof{figure}{Con la jaula de Faraday, la bolitas no tuvieron efecto alguno ya que la jaula se llevaba toda la carga, en cambio, con el recipiente de plastico las bolitas se repelieron por que la bolitas adquirian la misma carga}
    \label{fig:mesh10}
\end{Figuras}

	\subsection{Pantalla eléctrica}

\begin{Figuras}
	\centering
    \includegraphics[width=0.9\textwidth]{pantalla_c}
	\captionof{figure}{Con el capuchon no se aprecia ningun efecto visible al ser esta una cavidad encerrada conductora, es decir una jaula de Faraday}
    \label{fig:mesh11}
\end{Figuras}
.\\\\\\\\\\\\
	\subsection{Efecto de puntas}
	\begin{itemize}
		\item\textbf{Rehilete electrostático}


		\begin{Figuras}
			\centering
  			  \includegraphics[width=0.6\textwidth]{rehilete_c}
			\captionof{figure}{El rehilete empieza lentamente a girar y ganar aceleracion, esto debido a que las cargas se acumulan en las puntas ionizando el aire y creando viento electrico}
			    \label{fig:mesh12}
		\end{Figuras}

		\item\textbf{Mechon de cabello}

		\begin{Figuras}
			\centering
  			  \includegraphics[width=0.9\textwidth]{cabello_c}
			\captionof{figure}{El cabello empieza a parar sus puntas individualmente una a una, una vez mas, por que las cagas se concentran en las puntas, al estar los cabellos individuales cargados con la misma carga, se repelen}
			    \label{fig:mesh13}
		\end{Figuras}
.\\\\\\\\\\\\\\
		\item\textbf{Experiencia de la vela}

		\begin{Figuras}
			\centering
  			  \includegraphics[width=0.9\textwidth]{vela_c}
			\captionof{figure}{la vela se aleja de la punta de metal desde cualquier angulo. La combustion ioniza el aire provocando que la carga de la punta repela la carga del aire de la flama}
			    \label{fig:mesh14}
		\end{Figuras}

		\end{itemize}
\section{Cuestionario}

\begin{itemize}

	\item Defina:
		\begin{itemize}
		\item Densidad lineal de carga\\
Es la cantidad de carga eléctrica en una línea $\frac{C}{m}$
		\item Densidad superficial de carga\\
Es la cantidad de carga eléctrica en una superficie$\frac{C}{m^2}$
		\item Densidad volumetrica de carga\\
Es la cantidad de carga eléctrica en un  volumen$\frac{C}{m^3}$
		\end{itemize}.
	\item¿Que significado tienen las siguientes ecuaciones?\\
		\begin{itemize}
		\item $E=-\nabla V$\\
Define un campo electrico. Además, por el cálculo diferencial, se sabe que un campo cuyo rotacional es cero puede ser descrito mediante el gradiente de una función escalar, conocida como potencial eléctrico.
		\item $\nabla E= \frac{\rho}{\varepsilon_0}$\\
La divergencia de un campo electrico hacia una distribucion de carga
		\end{itemize}.
	\item ¿Es la superficie de un conductor una superficie equipotencial?\\
Si, porque cuando una parte de esta superficie se carga, lo hace en ambos de sus lados y a lo largo de toda la superficie de manera constante, ya que es una característica de los conductores, y por lo tanto también su potencial es constante.
	\item ¿Qué diferencia existe entr un electroscopio y un electrometro?\\
 El electroscopio es un aparato que permite detectar la presencia de carga eléctrica en un cuerpo e identificar el signo de la misma. Un electrómetro establece una diferencia de potencial entre la caja y la varilla con la lámina de oro (o la aguja de aluminio), esta es atraída por la pared del recipiente. La intensidad de la desviación puede servir para medir la diferencia de potencial entre ambas. 
\end{itemize}

\section{Conclusiones}

\subsection{Molina Escobar Carlos}

En la practica observamos diferentes objetos cargados con distintas geometrías y materiales que se comportaban de manera distinta, por lo tanto, podemos afirmar que el comportamiento de un objeto cargado con algún campo se ve afectado no solo por su carga, si no también, por su distribución, es decir, la geometría del objeto además del material.

\subsection{Parada Velazquez Kevin Osvaldo}
En el laboratorio, mediante la experimentación se pudo observar el funcionamiento de un electroscopio, así como determinar con él si un cuerpo está cargado y su signo de la carga. Se verificó que, en los conductores, la carga eléctrica se distribuye en la superficie exterior.
Se observó que, mediante el empleo de un electroscopio, en un conductor hueco, el campo eléctrico es nulo; así como también se observó el efecto con puntas al realizar los experimentos del rehilete, la vela y el mechón de cabellos con ayuda del Generador de Van de Graaff.


\subsection{Victoria Zúñiga Mario Alberto}
En al práctica dos podemos observar como el flujo de electrones es interrumpido por ciertos materiales en distintas situaciones .


\subsection{Santana Martínez Jesús Gerardo}

Con la práctica podemos concluir que los cuerpos dependen muchas veces de cargas y que estas estan interactuando a través de una carga positiva o negativa.
Además hay cuerpos que llamamos aislantes y muchas veces estos interfieren en el proceso de interacción de un cuerpo y como reacciona en el medio que se encuentra.

\subsection{Tovar Urrea Roberto Fausto}
En este experimento podemos concluir que la carga que adquieren los cuerpos conductores se distribuye alrededor de su superficie, lo que podemos considerar “un espacio perturbado”. Además, se demostró que en una cavidad encerrada por un conductor como el capuchón del electroscopio no se producen campos eléctricos en el interior.

\section{Bibliografia}
Serway, Raymond A. y John W. Jewett, Jr. Electricidad y magnetismo. Novena edición. ISBN: 978-607-522-490-9.\\
http://personales.upv.es/jquiles /prffi/electrostatica/ayuda/hlpdensidadcarga.htm

\end{multicols}
\printindex


\end{document}