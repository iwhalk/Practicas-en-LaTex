\documentclass[11pt]{article}


\title{Multímetro}
\author{2CV5\\Equipo 4\\\\Molina Escobar Carlos\\Victoria Zúñiga Mario Alberto\\Parada Velazquez Kevin Osvaldo}
\date{}

\usepackage[top=2cm, left=2cm, right=2cm, bottom=2cm]{geometry}
\usepackage{multicol}
\usepackage{graphicx}
\usepackage{caption}
\usepackage{enumitem}
\usepackage[spanish]{babel}
\usepackage{booktabs,chemformula}
\usepackage{verbatim}
\usepackage[document]{ragged2e}

\makeindex

\newenvironment{Figuras}
  {\par\medskip\noindent\minipage{\linewidth}}
  {\endminipage\par\medskip}


\graphicspath{ {c:/Users/iwhalk/Documents/LaTex/EyM_imagenes_3/} }


\begin{document}
	
\maketitle

\begin{abstract}
	El buen uso del multímetro es vital para cualquier estudiante de la carrera de Ingeniería de comunicaciones y electrónica, por lo que esta práctica tiene como objetivo de describir correctamente las características y el funcionamiento del multímetro. Para cumplir el objetivo es preciso realizar varias mediciones diferentes con el multímetro; estos elementos a medir son: resistencia, corrientes y tensión.
\end{abstract}


\begin{multicols}{2}
\section{Introduccion}
	\subsection{Multímetro: ¿Cómo se hace para medir?}
Un multímetro es un instrumento de medida. Un amperímetro mide
intensidad de corriente, un voltímetro mide la diferencia de potencial
entre dos puntos (voltaje), y un óhmetro mide resistencia. Un
multímetro combina estas funciones, y además algunas otras
adicionales, en un mismo instrumento.
Antes de empezar en los detalles sobre multímetros, es importante que
tengas las ideas claras de cómo se conectan en los circuitos. Las figuras 1 y 2 muestran un circuito antes y después de conectar un amperímetro:

\begin{Figuras}
	\centering
    \includegraphics[width=0.5\textwidth]{multimetro_1}
	\captionof{figure}{Para medir intensidad de corriente, el circuito debe abrirse para
permitir que el amperímetro se conecte en serie.}
    \label{fig:mesh1}
\end{Figuras}

Piensa en los cambios que debieras tener que hacer a un circuito
práctico para incluir el amperímetro. Para empezar, necesitas abrir el
circuito así el amperímetro puede conectarse en serie. Toda la corriente
que fluye en el punto del circuito a medir debe pasar a través del
amperímetro. Se supone que los instrumentos no alteran el
comportamiento del circuito, o al menos no significativamente, y se
deduce que un amperímetro debe tener una muy resistencia
interna.

\begin{Figuras}
	\centering
    \includegraphics[width=0.58\textwidth]{multimetro_2}
	\captionof{figure}{Los amperímetros suelen tener una resistencia propia muy baja.}
    \label{fig:mesh2}
\end{Figuras}


La figura 3  muestra el mismo circuito después de conectar un
voltímetro: esta vez, no necesitas abrir el circuito. El voltímetro es
conectado en paralelo entre los dos puntos donde se realiza la
medición. Dado que el voltímetro provee un camino paralelo, este
debería tomar en lo posible muy poca corriente. En otras palabras, un
voltímetro debe tener una resistencia muy alta.

\begin{Figuras}
	\centering
    \includegraphics[width=0.5\textwidth]{multimetro_3}
	\captionof{figure}{Para medir diferencia de potencial (voltaje), el circuito no se
cambia: el voltímetro se conecta en paralelo con el componente del
circuito.}
    \label{fig:mesh3}
\end{Figuras}

Un óhmetro no funciona con un circuito conectado
a la fuente de alimentación. Si quieres medir la
resistencia de un componente en particular, debes
quitarlo por completo del circuito y medirlo
separadamente, como muestra la figura 4:

\begin{Figuras}
	\centering
    \includegraphics[width=0.9\textwidth]{multimetro_4}
	\captionof{figure}{para medir resistencia, el componente debe ser
quitado por completo del circuito. Un óhmetro trabaja haciendo pasar una corriente a
través del componente que se quiere medir}
    \label{fig:mesh4}
\end{Figuras}

Los óhmetros funcionan haciendo pasar una pequeña corriente a través del componente y midiendo el
voltaje producido sobre el mismo. Si lo intentas hacer con el componente conectado en el circuito y
alimentado con una fuente, lo más probable es que el instrumento será dañado. La mayoría de los
multímetros tiene un fusible que ayuda a protegerlo ante estas imprudencias o uso inadecuado.

	\subsection{Multímetros digitales}
Los multímetros son diseñados y fabricados en serie por
ingenieros electrónicos. Aún el tipo más simple y más barato
puede incluir características las cuales es probable que no
uses. Los instrumentos digitales dan una salida numérica,
normalmente sobre un display de cristal líquido.

\begin{Figuras}
	\centering
    \includegraphics[width=0.9\textwidth]{multimetro_5}
	\captionof{figure}{}
    \label{fig:mesh5}
\end{Figuras}

La figura 5 muestra un multímetro de rango conmutado:
El conmutador central tiene muchas posiciones y tú debes
elegir la más apropiada para la medición que vas a realizar.
Si el mando es colocado para 20 VDC, por ejemplo, entonces
20 V es el máximo voltaje que puede ser medido. Este valor
es a veces llamado 20 V fsd, donde fsd es la abreviatura de
deflexión fondo de escala.
Para circuitos con fuentes de alimentación de hasta 20 V, lo
cual incluye todos los circuitos que probablemente
construyas, el rango de voltajes 20 VDC es el más útil. El
rango DC (direct current: corriente continua) está indicado
sobre el instrumento por:

Algunas veces, querrás medir voltajes más pequeños, y en
este caso, son usados los rangos de 2 V o 200 mV.
¿Qué significa DC? DC significa corriente continua.
Cualquier circuito que funciona con una fuente de tensión
estable, tal como una batería, la corriente siempre fluye en la
misma dirección. Cada circuito o proyecto construido en
este Curso trabaja de esta manera.
AC significa alternating current (corriente alterna). En
una lámpara eléctrica conectada en la red doméstica, la
corriente fluye primero de una manera, luego de otra. Esto
es, la corriente se invierte de polaridad, o se alterna, en
dirección.

El rango de voltaje AC es probable que casi no lo uses. Su símbolo en el instrumento es:

\subsubsection{¿Dónde están conectadas las dos puntas de prueba?}
La de color negro está siempre conectada dentro de la clavija o conector marcado como COM,
abreviatura de COMMON (común, masa, 0 V). La punta de color rojo es conectada dentro de la clavija
V, $\omega$ para medir con voltímetro u óhmetro o en la de A como amperímetro o miliamperímetro (para el
modelo que hay en el Taller). El conector o clavija de 10 A (10 amperios) sólo se usa cuando queremos
medir grandes corrientes de hasta 10 A y esto es muy raro en la mayoría de los circuitos que verás.

\section{Material y equipo}
	\begin{itemize}

		\item1  Multímetro digital.
		\item1 Multímetro análogo.
		\item1 Pila tipo “D” de 1.5v.
		\item1 Pila de valor desconocido.
		\item1 Protoboard.  
		\item3 Jumper (Macho-Macho).
		\item2 Jumper (Hembra - Macho).
		\item2 Cables “caimán - banana”
		\item2 Cables “banana – banana”
		\item1 Fuente de alimentación regulada.
		\item6 Resistencias de carbono de KΩ (los alumnos deben llevarlas).

	\end{itemize}

\section{Desarrollo experimental}
	\subsection{Reconocimiento del multímetro.}
	
	\subsection{Mediciones de resistencia (óhmetro).}

	\begin{itemize}

		\item Encienda el multímetro y coloque la perilla en ohms.
		\item Anote los valores utilizando el código de colores para resistencias.
		\item Mida las resistencias proporcionadas y anote el valor obtenido en la tabla.
		\item Compare el valor nominal con el valor medido.

	\end{itemize}

	\subsection{Mediciones de continuidad}

Utilizando un medidor en continuidad (buzzer) o en la escala de resistencia más baja identifique como está constituido un protoboard, dispositivo frecuentemente utilizado en la carrera para elaborar circuitos sencillos de las aplicaciones estudiadas en esta asignatura o en otras posteriores.

	\subsection{Mediciones de voltaje (voltímetro)}

	\begin{itemize}

		\item Medida de voltaje de pilas/baterías.
		\item Coloque la perilla del multímetro en volts y ubique correctamente el selector de tipo de corriente (ca/cd) en cd.
		\item Mida el voltaje de las pilas.
		\item Utilice la fuente regulada, ubique las salidas de corriente directa y realice 3 (tres) mediciones diferentes respetando la polaridad.

	\end{itemize}


	\subsection{Mediciones de voltaje de corriente alterna (vóltmetro).}

	\begin{itemize}

		\item Mediciones de voltaje de corriente alterna (vóltmetro).
		\item Ubique un contacto como el mostrador en la figura en su mesa de trabajo.
		\item Mida el voltaje entre salidas.
		\item Identifique en el contacto cual es la fase, el neutro y la tierra física.
	\end{itemize}

	\subsection{Mediciones de intensidad de corriente eléctrica continúa (amperímetro).}

	\begin{itemize}

		\item Arme el siguiente circuito.
			\begin{Figuras}
				\centering
    				\includegraphics[width=0.9\textwidth]{multimetro_6}
				\captionof{figure}{}
			    	\label{}
			\end{Figuras}
		\item Seleccione la escala de medición adecuada y el selector de cd/ca colóquelo en cd.
		\item Mida la corriente.
	\end{itemize}

\section{Resultados}
	\subsection{Reconocimiento del multímetro.}
	
\begin{Figuras}
	\centering
    \includegraphics[width=0.4\textwidth]{multimetro_7}
	\captionof{figure}{Multimetro marca PeakTech modelo  2005}
    \label{fig:mesh7}
\end{Figuras}

Este multímetro simplemente se enciende con un botón. El multímetro tiene varios selectores para medir corriente continua y alterna, así como, tensión de la misma manera, además, tiene varias escalas que permiten ir del orden de los milis hasta los megas.

\begin{Figuras}
	\centering
    \includegraphics[width=0.9\textwidth]{multimetro_8}
	\captionof{figure}{La medicion de los resistores se realiza en paralelo}
    \label{fig:mesh8}
\end{Figuras}

	\subsection{Mediciones de resistencia (óhmetro).}

\begin{Figuras}
	\centering
    \includegraphics[width=1.2\textwidth]{multimetro_9}
\end{Figuras}

	\subsection{Mediciones de continuidad}

\begin{Figuras}
	\centering
    \includegraphics[width=0.9\textwidth]{multimetro_10}
	\captionof{figure}{La protoboard esta contituida de filas y columnas: las filas conforman nodos conectados entre si y las columnas no, ademas, estan las barras laterales que forman 2 nodos separados cada una}
    \label{fig:mesh9}
\end{Figuras}
	
	\subsection{Mediciones de voltaje (voltímetro).}

\begin{Figuras}
	\centering
    \includegraphics[width=0.9\textwidth]{multimetro_11}
	\captionof{figure}{Para la primera pila(izquierda) obtuvimos 1.06V  y para la segunda(Derecha) 6.18V}
    \label{fig:mesh10}
\end{Figuras}

Asi mismo con la fuente regulada  hicimos 3 mediciones diferentes de 1.48V, 4.38V y 14.57V

	\subsection{Mediciones de voltaje de corriente alterna (vóltmetro).}

\begin{Figuras}
	\centering
    \includegraphics[width=0.9\textwidth]{multimetro_12}
	\captionof{figure}{Realizamos 3 mediciones en 3 contactos diferentes: 125.9V, 25.2V y 127.1V respectivamente}
    \label{fig:mesh11}
\end{Figuras}

	\subsection{Mediciones de intensidad de corriente eléctrica continúa (amperímetro).}

\begin{Figuras}
	\centering
    \includegraphics[width=0.9\textwidth]{multimetro_13}
	\captionof{figure}{Las mediciones de corriente en un circuito se tiene que hacer con el multimetro en serie con la fuente	 y el elemento a medir. La corriente en este caso nos dio 10.21 mA}
    \label{fig:mesh12}
\end{Figuras}

Por la ley de Ohm tenemos:

$$I= \frac{V}{R}$$

Sustituyendo:
$$I= \frac{10V}{1K\Omega} = 10mA$$

Por lo que nos queda un valor muy aproximado al del multímetro 

\section{Conclusiones}

\subsection{Molina Escobar Carlos}

El multímetro es una herramienta indispensable para cualquier estudiante y profesional involucrado con la electrónica, ya que este puede realizar mediciones bastante precisas de varias unidades, como lo son, la corriente, tensión, capacitancia resistencia entere otras; a esto se le agrega la capacidad de trabajar en corriente continua y directa.

\subsection{Parada Velazquez Kevin Osvaldo}
Como sabemos el multímetro es un instrumento que permite medir diferentes magnitudes eléctricas. Nos ayuda para prevenir algún corto o falla dentro de nuestra instalación eléctrica o circuito. Y saber las propiedades que tiene dicha cantidad de energía para poder usar el tipo de resistencias adecuadas. 


\subsection{Victoria Zúñiga Mario Alberto}
El multímetro es un aparato de medición con el cual podemos detectar problemas de corriente en sistemas electrónicos y eléctricos o incluso solos para verificar y analizar mejor el comportamiento de la corriente eléctrica en distintas situaciones.


\section{Bibliografia}
Serway, Raymond A. y John W. Jewett, Jr. Electricidad y magnetismo. Novena edición. ISBN: 978-607-522-490-9.\\
http://roble.pntic.mec.es/jlop0164/archivos/multimetro.pdf

\end{multicols}


\end{document}