\documentclass[11pt]{article}

\usepackage[top=2cm, left=2cm, right=2cm, bottom=2cm]{geometry}
\usepackage{multicol}
\usepackage{graphicx}
\usepackage{caption}
\usepackage{enumitem}
\usepackage[spanish]{babel}

\title{Distribución de las Cargas Électricas en los Conductores}
\author{2CV5\\Equipo 4\\\\Molina Escobar Carlos\\Victoria Zúñiga Mario Alberto\\Parada Velazquez Kevin Osvaldo\\Santana Martínez Jesús Gerardo\\Tovar Urrea Roberto Fausto}
\date{}

\newenvironment{Figuras}
  {\par\medskip\noindent\minipage{\linewidth}}
  {\endminipage\par\medskip}

%\graphicspath{ {c:/Users/iwhalk/Documents/LaTex/Ondas_p2/} }

\begin{document}

\begin{abstract}
	El péndulo simple (también llamado péndulo matemático o péndulo ideal) es un sistema idealizado constituido por una partícula de masa m que está suspendida de un punto fijo o mediante un hilo inextensible y sin peso. Naturalmente es imposible la realización práctica de un péndulo simple, pero si es accesible a la teoría. El péndulo simple o matemático se denomina así en contraposición a los péndulos reales, compuestos o físicos, únicos que pueden construirse. El péndulo de torsión consiste en un hilo o alambre de sección recta circular suspendido verticalmente, con su extremo superior fijo y de cuyo extremo inferior se cuelga un cuerpo de momento de inercia I conocido o fácil de calcular (disco o cilindro). Cualquier movimiento puede descomponerse como combinación de movimientos lineales y de rotación. Un péndulo físico o péndulo compuesto es cualquier cuerpo rígido que pueda oscilar libremente en el campo gravitatorio alrededor de un eje horizontal fijo, que no pasa por su centro de masa.
\end{abstract}


\begin{multicols}{2}
asd
\end{multicols}

\end{document}