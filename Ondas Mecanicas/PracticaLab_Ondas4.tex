\documentclass[11pt]{article}

\usepackage[top=2cm, left=2cm, right=2cm, bottom=2cm]{geometry}
\usepackage{multicol}
\usepackage{graphicx}
\usepackage{caption}
\usepackage{enumitem}
\usepackage[spanish]{babel}
\usepackage{gensymb}

\title{Movimiento Ondulatorio de una Cuerda}
\author{Molina Escobar Carlos\\Benitez Alanis Rodrigo Moctezuma\\Ortiz Benhumea Jose Rogelio}
\date{}

\newenvironment{Figuras}
  {\par\medskip\noindent\minipage{\linewidth}}
  {\endminipage\par\medskip}

\graphicspath{ {c:/Users/iwhalk/Documents/LaTex/Ondas_p3/} }

\begin{document}

\maketitle

\begin{abstract}

	 Movimiento ondulatorio en una cuerda, es el proceso por el que se propaga energía de un lugar a otro sin transferencia de materia, mediante ondas mecánicas o electromagnéticas.
	Podemos observar ejemplos de movimiento ondulatorio en la vida diaria: el sonido producido en la laringe de los animales y de los hombres que permite la comunicación entre los individuos de la misma especie, las ondas producidas cuando se lanza una piedra a un estanque, las ondas electromagnéticas producidas por emisoras de radio y televisión, etc.
	Antes de que Hertz realizara sus experimentos para producir por primera vez ondas electromagnéticas, su existencia había sido predicha por Maxwell como resultado de un análisis cuidadoso de las ecuaciones del campo electromagnético. El gran volumen de información que se ha acumulado sobre las ondas electromagnéticas (cómo se producen, propagan, y absorben) ha posibilitado el mundo de las comunicaciones que conocemos hoy en día.
	Aunque el mecanismo físico puede ser diferente para los distintos movimientos ondulatorios, todos ellos tienen una característica común, son situaciones producidas en un punto del espacio, que se propagan a través de este y se reciben en otro punto.

\end{abstract}

\begin{multicols}{2}

\section{Objetivo}
	Observar como la fuerza se traduce en energía al crear ondulaciones dentro de un medio, en este caso una cuerda de rigidez muy tenue, y así dar a conocer el régimen que tienen las ondas dentro del medio físico.
\section{Téoria}

		\subsection{Movimiento Ondulatorio.}
				
				Cuando deseas transmitir energía a un objeto distante puedes hacerlo ayudándote de otro cuerpo, sin necesidad de que tengan que colisionar.  Es el caso del movimiento ondulatorio, en el que se produce transmisión de energía sin transmisión de materia.

				El movimiento ondulatorio o movimiento de propagación de onda se define como una perturbación que se propaga de un punto a otro sin que exista transporte neto de materia, pero sí transmisión de energía.

				Cuando la perturbación se propaga a través de un medio material, se denomina onda mecánica, por ejemplo, las ondas generadas en la cuerda de una guitarra o sobre la superficie de un lago. Las únicas ondas que se pueden propagar por el vacío son las ondas electromagnéticas (espectro), como por ejemplo la radiación solar, los rayos X o la luz visible.

		\subsection{Características de las ondas.}

				\begin{itemize}
				
						\item Se hace necesario un foco emisor o fuente que actúe como origen de la perturbación. La energía del foco es transmitida al medio de propagación en sus inmediaciones
						\item Debe existir un medio de propagación que, a medida que es atravesado por la perturbación, experimenta una variación temal y reversible en alguna de sus propiedades físicas. Dicho medio, material o no, sirve de soporte a la transmisión del movimiento ondulatorio, pero no es transportado sí mismo
						\item Cada punto del medio transmite la perturbación a los puntos vecinos. De esta manera, podemos decir que el fenómeno onduorio es una forma cooperativa de propagación de la energía en la que esta se transmite entre el foco y los puntos alcanzados
						\item A medida que la perturbación se propaga la onda se amortigua. Esta amortiguación se debe al reparto de energía que se vroduciendo a medida que la perturbación viaja alcanzando un espacio cada vez mayor, pero también se debe a otros factores como el grado de elasticidad del medio o el pose rozamiento entre partículas
						\item Existe un retardo entre el movimiento ondulatorio y el momento en que los puntos más lejanos son alcanzados. Esto pone manifiesto una velocidad finita de propagación de las ondas.
						\item La onda no es un ente material, pero si una entidad física real ya que transporta energía e interacciona con la materia

				
				\end{itemize}

				Las ondas armónicas son aquellas en las que la perturbación que las genera describe un movimiento armónico simple.
				Existen varios criterios para clasificar las ondas. Basándonos en las direcciones en las que se propagan se pueden distinguir ondas unidimensionales, bidimensionales o tridimensionales. 

				\subsubsection{Según la dirección en la que se propaga la energía se clasifican en:}

						\begin{itemize}

							\item Ondas transversales: se caracterizan porque la dirección de propagación de la energía es perpendicular a la dirección en la que oscilan las partículas del medio material por el que se propagan.
							\item Ondas longitudinales: en ellas la dirección de propagación coincide con la dirección en la que oscilan las partículas del medio por el que se propaga. El sonido es una onda longitudinal.

						\end{itemize}

\section{Desarrollo Experimental}

		\subsection{Material y equipo}

			\begin{itemize}

				\item 1 cuerda combinada.	
				\item 1 soporte para fijar la cuerda.
				\item 1 cuerda de goma con extremos de cuerda de algodón.
				\item 1 pelota con vástago para fijar.
				\item 1 motor de 220 V AC con reductor de velocidad de 1:10.
				\item 1 pinzas de mesa o tripie.
				\item 1 base para el motor.	
				\item 1 estroboscopio.
				\item 1 Rueda acanalada (Hoffmann).
				\item 1 balanza granataria o báscula electrónica.

			\end{itemize}

		\section{Desarrollo}

				\subsubsection{Determinación de la velocidad de fase.}

					Como primer paso se montó el motor y la varilla en las orillas de dos mesas diferentes, posteriormente se fijó la cuerda del extremo de la varilla hacía el motor, con una evidente inclinación de $45^o$.

					El cilindro que servía de base donde se fijaba la cuerda al motor, presentaba un ranurado senoidal por donde iba a dar el trayecto la cuerda, la cual al vibrar repetía la ondulación.

					Como primer ensayo, se perturbo el sistema accionando el movimiento del motor y encontrando dos ondulaciones dentro de la cuerda, con ayuda de un metro se marcó un periodo completo y con el estroboscopio se proyectó el haz de luz que este emanaba a cada cierto tiempo sobre la cuerda, con ayuda de una perilla que tenía en la parte superior se regulando su frecuencia de modo que al reflejarlo sobre la cuerda el fantasma de su recorrido se notara claramente, casi como si la cuerda se quedara estática. La frecuencia registrada en la pantalla que se encontraba en la parte posterior y centímetros arriba de la perilla variable del mismo era la cual se debía registrar para hacer los cálculos posteriores.

				\subsubsection{Determinación de la velocidad de fase en función de la fuerza aplicada a la cuerda.}

					Para esta práctica se hizo lo mismo que la anterior, el cambio que se debía hacer era únicamente en las perturbaciones de la cuerda para generar más oscilaciones, lo anterior se lograba al variar la velocidad del motor. Se repetía el proceso de general el haz de luz sobre la cuerda, regulado la frecuencia para emular la de la cuerda y a su vez midiendo el ciclo completo de oscilación con ayuda del metro, registrando todo en el practicario para los cálculos posteriores.

					\begin{Figuras}
						\centering
					    \includegraphics[width=1.1\textwidth]{superposicion_gra}
						\captionof{figure}{Grafica de $\lambda$ contra  $\upsilon$}
					    \label{fig:mesh1}
					\end{Figuras}

					Por errores de medición, no fue posible conseguir una buena grafica que represente la realidad.

\section{Conclusiones}
	
	\subsection{Benitez Alanis Rodrigo Moctezuma}
	
		En este experimento hemos podido observar el fenómeno de trasmisión de energía por un medio completamente mecánico, a través de la generación de ondas senoidales por donde se emula un sistema de comunicación. La perturbación que presenta todo sistema depende de la fuerza con la que se mande el “mensaje” hacia un medio (receptor), entre más perturbaciones genere el transmisor en el medio, será mayor la intensidad con la que se trasmita dicha fuerza o imagen de la señal.

		 Además de eso, pude observar que a medida que la fuerza aumenta también lo hacía la manera en la que la onda recorre cierto trayecto, es decir que la fuerza será equiparable a la velocidad con la que se propague esta perturbación sobre el medio. Hay que destacar que la fuerza también le dará mayor vitalidad a la duración de la perturbación, con ello quiero decir que, en el momento en que se dejaba de dar marcha al motor, dependiendo de la fuerza con la que este había iniciado a perturbar el sistema, la cuerda seguía vibrando instancias después de que se dejara de operar el movimiento. Quizá esto lo pueda explicar desde el punto de vista musical, cuando toco guitarra hay dos factores importantes para acentúa una nota musical, la cual es una técnica del plumilleo: si deseo que la nota se escuche suave y tenga una duración mínima, lo que debo hacer es puntear la cuerda de modo tal que no le implique una fuerza desmedida que haga que oscile demás y sobre pase el sonido tenue que deseo emular, de este modo no solo se escucha un sonido suave, delgado, si no que a su vez la armonía que sale de la nota se apaga en cuestión de segundos; por el contrario, cuando busco un sonido más agresivo y que tenga mayor presencia,  es decir, que tenga una duración que probablemente pueda llegar cerca del minuto y medio, al momento de puntear la nota debo imprimirle más fuerza y quizá continuar en momentos cruciales para que no pierda brillo (tipo movimiento armónico forzado), pero respetando el tiempo de vibración (frecuencia) de la nota para evitar hacer un sobre sonido por el rose constante de la plumilla con la cuerda. En otras palabras, hay una remanencia de la fuerza que se trasmite a raíz de las ondas, remanencia que va decreciendo a medida que pasa el tiempo debido a factores como el amortiguamiento mismo.
		 
		Una onda es sin duda un medio por el cual se puede trasmitir información, en este caso fuerza y energía, entre otras cosas entran a detalle en la carrera de comunicaciones ya que es donde se habla de una señal que posteriormente se traduce en información. El alcance de una onda depende de la fuerza con la que esta se emane a un punto especifico y (como anteriormente vi en la materia de campos y ondas) el remanente es el que llega al resto de los demás receptores en espera de la señal. Pero también tiene una fuente energética completa, ya que en la carrera de electricidad se habla no de señales en sí, si no más bien del trasporte de energía entre terminales que por supuesto lo hace a través de pulsaciones ondulares de un emisor a un receptor. 

		Es ahí cuando nos damos cuenta del alcance y las aplicaciones de este ejercicio que da por sentado que las ondas se encuentran en todo nuestro entorno, desde señales eléctricas, señales informáticas e inclusive señales que emiten nuestro cerebro para mantener conectado y coordinado a nuestro resto del cuerpo. Todo se trasmite mediante ondas, ya sean mecánicas o eléctricas y, por que por supuesto, satisfacen el principio del movimiento armónico amortiguado puesto que hay una ondulación en el espacio tiempo como lo hace el sistema presentado por Hook a la hora de usar un resorte amarrado a una masa, se traduce en un movimiento adulatorio. 

	\subsection{Molina Escobar Carlos}

		El movimiento ondulatorio en la cuerda supone que determinar la rapidez de propagación de una onda transversal es posible determinando los parámetros que influyen directamente. Entonces conforme la longitud de onda fuera disminuyendo, aumentaría la frecuencia, que en nuestro caso no se vio claramente debido a torpes errores de medición.

	\subsection{Ortiz Benhumea Jose Rogelio}

		A medida que las longitudes de onda fueron disminuyendo sus valores debido a que el torno aumentaba su rapidez de giro, las frecuencias fueron haciéndose de mayor magnitud y las velocidades (teóricamente se dice que deben salir casi iguales), tuvieron una diferencia de 2 a 3 $\frac{m}{s}$ de velocidad comprobando de esta manera que la ecuación del efecto concierto de cierta mayor contribución de frecuencia menor será la longitud de onda y la menor contribución de frecuencia mayor será el alcance de la Honda entre una cresta y otra pero su velocidad será casi siempre la misa.

\section{Bibliografia}

	\textit{https://www.fisicalab.com/apartado/que-son-las-ondas}
	\textit{http://www2.montes.upm.es/dptos/digfa/cfisica/ondas/ondasintro.html}


\end{multicols}

\end{document}